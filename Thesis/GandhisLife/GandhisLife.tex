\chapter{Gandhi's Life}
To answer the first part of the question of how Gandhi�s spirit and dream looked like in detail, it is important to have a closer look at Gandhi�s life. This is necessary to draw similarities between the life he experienced and his principles later on. Gandhi�s life, especially his political engagement, was marked by various events. That is why describing his life in only a few pages would not sufficient to show his life in every facet, but this chapter tries to scribe the main events to give an impression about the important steps of Gandhi�s life. 

\section{Gandhi�s Childhood and Studies in London}
Mohandas Karamchand Gandhi, as he was originally called, was born in Porbandar on the 2nd of October 1869. Porbandar is a town on a peninsular that belonged to the federal state of India \cite{hoepken2001gewaltfreiheit} and it is located in the west of India. His family was part of the third of four main castes, which is called the Vaishya. This is  a caste for retailers, because Gandhi�s ancestors used to be retailers \cite{hoepken2001gewaltfreiheit}. Every caste is divided in more little castes below. His family was therefore also part of the Banias. As a member of this cast the family was wealthy and well-educated.

Gandhi and his family believed in the Vaishnavismus, a main principal direction of the Hinduism \cite{hoepken2001gewaltfreiheit}, in which they believe in the highest god called Vishnu. Gandhi�s mum influenced the development of Gandhi�s character very much. She was highly religious \cite{hoepken2001gewaltfreiheit} and fasting was an extremely important part of her life \cite{gandhi1930}, what will play a major role in Gandhi�s life later on, too.
India at that time was fragmented into several city states. Gandhi�s father was the premiere minister of Porbandar. \cite{hoepken2001gewaltfreiheit} Also his father and his way of living had a big effect on Gandhi�s beliefs and principles. Gandhi�s father did never have the goal to amass money and lived his life in simplicity \cite{gandhi1930}, what will become part of Gandhi�s life later, too. 

Gandhi was, according to his own statement, never a very good student and his grades were constantly average. \cite{gandhi1930} He also described himself as "very timid" \cite{gandhi1930} and he obviated social interactions when he was younger. In 1888 Gandhi moved to London to start his studies of law. \cite{hoepken2001gewaltfreiheit} After he had finished his studies he returned to India, where he worked in a chancery. 

\section{Gandhi�s Life in South-Africa }
In 1893 Gandhi�s chancery sent him to South-Africa to take over a case. Living in South-Africa he was confronted with racism against his Indian fellow countryman. There were three groups of Indian people living in South-Africa at that time: Muslim traders, Parses and Hindus. Hindus in South-Africa were mostly outreach worker who worked at tea and sugar plantations for a limited period of time, regularly for four years. \cite{hoepken2001gewaltfreiheit}

Gandhi personally experienced racism, e.g. he was thrown out of a train, although he had a valid ticket. \cite{hungerb�hler1983pioniere} As a lawyer he also got in many other situations, where being Indian provided him a big disadvantage and he consequentially had many problems concerning his ethical background. His opponents and even the judges didn�t take him serious and qualified him as a "Kuli" (way of how British called Indians in South-Africa). Other discriminations he experienced: Gandhi was not allowed to stay in several "better" hotels and once he was chased of a post chaise, followed by being beaten by multiple white men. 

The situation for all Indians in South-Africa was serious and so a delegation of politically interested Indians asked Gandhi to help his fellow countryman to get them out of their depressed situation. Although Gandhi had already planned to leave South-Africa to return to India, he decided to face his responsibility as a lawyer to help his Indian fellow countrymen. At that point of his life, Gandhi had sorrows concerning his character. He was not sure whether he was able to move masses, unfounded sorrows how we know now. \cite{hungerb�hler1983pioniere}

There were indeed not many rights for Indian people at this time; Indians were suppressed by both: Boers and British. It was allowed to beat Indians without being punished, furthermore Indians were taken to prison without any court decision or to ghettos where they were not allowed to leave. \cite{hungerb�hler1983pioniere}

Gandhi ultimately stayed in South-Africa for 20 years and achieved remarkable improvements for the Indian people there. He was responsible for many campaigns e.g. against the abolishment of the Indian right to vote and the taxation of the Indian outreach worker. First step in his fight for the Indians he collected 10 000 signs and ensured the interests of the Indians by founding the Natal Indian Congress in 1894 \cite{hungerb�hler1983pioniere}, an organization open for every Indian. As a result, the Kulis with no rights became a strong mass of people. Gandhi at that time developed to be a very important contact person for all Indians living in South-Africa and he was taking this responsibility enormous serious. When he returned to Indian for personal reasons, he still campaigned for his concerns and never got tired of repeating how unacceptable the situation in South-Africa was for all Indians. \cite{gandhi1930} Returning back to South-Africa, Gandhi brought with his wife and children, what showed that he really had the plan to stay longer. \cite{hungerb�hler1983pioniere}

It is remarkable that during fighting between Boers and Britain, he was always more on the side of Britain. So he also helped the British in the Boer War, which lasted from 1899-1902, as a paramedic. \cite{gandhi1930} This can be explained by the fact that Gandhi had lived in London for a long time and may felt closer to the British then the Boers.

Apart from Gandhi�s fight against the Indian discrimination, "he was constantly searching for a form of life, that connects the unity of faith and doing in an ideal way" \cite{hoepken2001gewaltfreiheit}. He toke the vow of Brahmacharya, what means to live in chastity, with the goal to built a community with religious foundations based on physical work for the community life to lead to the fundamental goals of such a community: equality and simplicity. \cite{hoepken2001gewaltfreiheit} This is how phoenix-settlements were founded in 1904, followed by the Tolsoi-Farm. At that time Gandhi quit his job as a lawyer to focus on his plans.

Two years later, in August 1906, a governmental journal published a law called the Black Ordinance that forced every Indian older than 8 years to register via fingerprints when coming to South-Africa. Gandhi announced his passive resistance, accompanied by his raising conviction of the nonviolence as a method in political conflict. \cite{hoepken2001gewaltfreiheit} He named his resistance movement Satyagraha (Hindi: hold on to the truth), whose word meaning will be explained in a following chapter. The new law about the registration led Gandhi and his supporter burn their registration cards 1908 which can be seen as the beginning of their civil disobedience. This was the time when he was arrested for the first time; follow by many other prison sentences over the following years. \cite{hoepken2001gewaltfreiheit} In 1909 Gandhi, still living in South-Africa, wrote his book called Hind Swaraj (Hindi: self-government of India) or Indian Home Rule, with which he started expanding his ideas on his home country itself. Gandhi constantly criticized the way the British people in India forces the Indian to live in a materialistic way what automatically meant the loss of traditions. \cite{hoepken2001gewaltfreiheit} He advised his home country to abandon all that they have learnt for the last 50 years: the railway, the telegraph, the hospitals and the medicine. Gandhi�s wish of a self-governmental India at that point was not about undertaking the governmental power of India by Indian elite, but to let India go the way of self-discovery. In 1912 he made a sign by giving away all his private property, possibly to speak out against materialism again.

Gandhi, still in South-Africa, continued fighting for his beliefs. When in 1913 the government published new laws concerning the immigration of Indians and additionally declared marriages made in a Hindu way as not valid, the Satyagraha continued their work. \cite{hoepken2001gewaltfreiheit} Gandhi�s men passed the border, where they were being beaten down by the police immediately, without reacting violent themselves. Gandhi was arrested again but had to be released because of increasingly louder protests coming from England and India.

The year 1914 was an important year for Gandhi�s fight in South-Africa, because an agreement was made between Gandhi and Smut, the home secretary at that time. As a result, the discriminating laws of registration were abandoned and so the Satyagraha ended with success. \cite{hoepken2001gewaltfreiheit} In 1915 he finally moved back to India, many years later than originally planned but in the moral certainty of having helped the Indians in South-Africa. 

\section{Gandhi Back in India}
Although Gandhi had been living in South-Africa for a very long time, he was always conscious about how life for the Indian people in India looked like. India at that time, as a colony of Britain, was under full control of the British Empire. Gandhi, as mentioned before, even appealed to the Indians to not make themselves independent of the British when he was still in South-Africa. Returning back to India it was time to implement his ideas to help the Indians to become an independent state.

\paragraph{The Situations of the Farmers and Gandhi�s Ideal Conception of them Living in Ashrams}


