\chapter{Gandhi's Life}
To answer the first part of the question of how Gandhi's spirit and dream looked like in detail, it is important to have a closer look at Gandhi's life. This is necessary to draw similarities between the life he experienced and his principles later on. Gandhi's life, especially his political engagement, was marked by various events. That is why describing his life in only a few pages would not sufficient to show his life in every facet, but this chapter tries to scribe the main events to give an impression about the important steps of Gandhi's life. 

\section{Gandhi's Childhood and Studies in London}
Mohandas Karamchand Gandhi, as he was originally called, was born in Porbandar on the 2nd of October 1869. Porbandar is a town on a peninsular that belonged to the federal state of India \cite{hoepken2001gewaltfreiheit} and it is located in the west of India. His family was part of the third of four main castes, which is called the Vaishya. This is  a caste for retailers, because Gandhi's ancestors used to be retailers \cite{hoepken2001gewaltfreiheit}. Every caste is divided in more little castes below. His family was therefore also part of the Banias. As a member of this cast the family was wealthy and well-educated.

Gandhi and his family believed in the Vaishnavismus, a main principal direction of the Hinduism \cite{hoepken2001gewaltfreiheit}, in which they believe in the highest god called Vishnu. Gandhi's mum influenced the development of Gandhi's character very much. She was highly religious \cite{hoepken2001gewaltfreiheit} and fasting was an extremely important part of her life \cite{gandhi1930}, what will play a major role in Gandhi's life later on, too.
India at that time was fragmented into several city states. Gandhi's father was the premiere minister of Porbandar. \cite{hoepken2001gewaltfreiheit} Also his father and his way of living had a big effect on Gandhi's beliefs and principles. Gandhi's father did never have the goal to amass money and lived his life in simplicity \cite{gandhi1930}, what will become part of Gandhi's life later, too. 

Gandhi was, according to his own statement, never a very good student and his grades were constantly average. \cite{gandhi1930} He also described himself as "very timid" \cite{gandhi1930} and he obviated social interactions when he was younger. In 1888 Gandhi moved to London to start his studies of law. \cite{hoepken2001gewaltfreiheit} After he had finished his studies he returned to India, where he worked in a chancery. 

\section{Gandhi's Life in South-Africa }
In 1893 Gandhi's chancery sent him to South-Africa to take over a case. Living in South-Africa he was confronted with racism against his Indian fellow countryman. There were three groups of Indian people living in South-Africa at that time: Muslim traders, Parses and Hindus. Hindus in South-Africa were mostly outreach worker who worked at tea and sugar plantations for a limited period of time, regularly for four years. \cite{hoepken2001gewaltfreiheit}

Gandhi personally experienced racism, e.g. he was thrown out of a train, although he had a valid ticket. \cite{hungerb�hler1983pioniere} As a lawyer he also got in many other situations, where being Indian provided him a big disadvantage and he consequentially had many problems concerning his ethical background. His opponents and even the judges didn't take him serious and qualified him as a "Kuli" (way of how British called Indians in South-Africa). Other discriminations he experienced: Gandhi was not allowed to stay in several "better" hotels and once he was chased of a post chaise, followed by being beaten by multiple white men. 

The situation for all Indians in South-Africa was serious and so a delegation of politically interested Indians asked Gandhi to help his fellow countryman to get them out of their depressed situation. Although Gandhi had already planned to leave South-Africa to return to India, he decided to face his responsibility as a lawyer to help his Indian fellow countrymen. At that point of his life, Gandhi had sorrows concerning his character. He was not sure whether he was able to move masses, unfounded sorrows how we know now. \cite{hungerb�hler1983pioniere}

There were indeed not many rights for Indian people at this time; Indians were suppressed by both: Boers and British. It was allowed to beat Indians without being punished, furthermore Indians were taken to prison without any court decision or to ghettos where they were not allowed to leave. \cite{hungerb�hler1983pioniere}

Gandhi ultimately stayed in South-Africa for 20 years and achieved remarkable improvements for the Indian people there. He was responsible for many campaigns e.g. against the abolishment of the Indian right to vote and the taxation of the Indian outreach worker. First step in his fight for the Indians he collected 10 000 signs and ensured the interests of the Indians by founding the Natal Indian Congress in 1894 \cite{hungerb�hler1983pioniere}, an organization open for every Indian. As a result, the Kulis with no rights became a strong mass of people. Gandhi at that time developed to be a very important contact person for all Indians living in South-Africa and he was taking this responsibility enormous serious. When he returned to Indian for personal reasons, he still campaigned for his concerns and never got tired of repeating how unacceptable the situation in South-Africa was for all Indians. \cite{gandhi1930} Returning back to South-Africa, Gandhi brought with his wife and children, what showed that he really had the plan to stay longer. \cite{hungerb�hler1983pioniere}

It is remarkable that during fighting between Boers and Britain, he was always more on the side of Britain. So he also helped the British in the Boer War, which lasted from 1899-1902, as a paramedic. \cite{gandhi1930} This can be explained by the fact that Gandhi had lived in London for a long time and may felt closer to the British then the Boers.

Apart from Gandhi's fight against the Indian discrimination, "he was constantly searching for a form of life, that connects the unity of faith and doing in an ideal way" \cite{hoepken2001gewaltfreiheit}. He toke the vow of Brahmacharya, what means to live in chastity, with the goal to built a community with religious foundations based on physical work for the community life to lead to the fundamental goals of such a community: equality and simplicity. \cite{hoepken2001gewaltfreiheit} This is how phoenix-settlements were founded in 1904, followed by the Tolsoi-Farm. At that time Gandhi quit his job as a lawyer to focus on his plans.

Two years later, in August 1906, a governmental journal published a law called the Black Ordinance that forced every Indian older than 8 years to register via fingerprints when coming to South-Africa. Gandhi announced his passive resistance, accompanied by his raising conviction of the nonviolence as a method in political conflict. \cite{hoepken2001gewaltfreiheit} He named his resistance movement Satyagraha (Hindi: hold on to the truth), whose word meaning will be explained in a following chapter. The new law about the registration led Gandhi and his supporter burn their registration cards 1908 which can be seen as the beginning of their civil disobedience. This was the time when he was arrested for the first time; follow by many other prison sentences over the following years. \cite{hoepken2001gewaltfreiheit} In 1909 Gandhi, still living in South-Africa, wrote his book called Hind Swaraj (Hindi: self-government of India) or Indian Home Rule, with which he started expanding his ideas on his home country itself. Gandhi constantly criticized the way the British people in India forces the Indian to live in a materialistic way what automatically meant the loss of traditions. \cite{hoepken2001gewaltfreiheit} He advised his home country to abandon all that they have learnt for the last 50 years: the railway, the telegraph, the hospitals and the medicine. Gandhi's wish of a self-governmental India at that point was not about undertaking the governmental power of India by Indian elite, but to let India go the way of self-discovery. In 1912 he made a sign by giving away all his private property, possibly to speak out against materialism again.

Gandhi, still in South-Africa, continued fighting for his beliefs. When in 1913 the government published new laws concerning the immigration of Indians and additionally declared marriages made in a Hindu way as not valid, the Satyagraha continued their work. \cite{hoepken2001gewaltfreiheit} Gandhi's men passed the border, where they were being beaten down by the police immediately, without reacting violent themselves. Gandhi was arrested again but had to be released because of increasingly louder protests coming from England and India.

The year 1914 was an important year for Gandhi's fight in South-Africa, because an agreement was made between Gandhi and Smut, the home secretary at that time. As a result, the discriminating laws of registration were abandoned and so the Satyagraha ended with success. \cite{hoepken2001gewaltfreiheit} In 1915 he finally moved back to India, many years later than originally planned but in the moral certainty of having helped the Indians in South-Africa. 

\section{Gandhi Back in India}
Although Gandhi had been living in South-Africa for a very long time, he was always conscious about how life for the Indian people in India looked like. India at that time, as a colony of Britain, was under full control of the British Empire. Gandhi, as mentioned before, even appealed to the Indians to not make themselves independent of the British when he was still in South-Africa. Returning back to India it was time to implement his ideas to help the Indians to become an independent state.

\paragraph{The Situations of the Farmers and Gandhi's Ideal Conception of them Living in Ashrams}
Gandhi allowed himself one year to get an authentic impression of India's political, economic and social areas. He was from now on called Mahatma, what means "big soul" \cite{hoepken2001gewaltfreiheit}, to express honor and gratefulness for what he had already done to help the Indian people. 

Gandhi experienced an India that suffered hardly from being colonised. Being under full British control meant for many Indians: living as farmers in big poverty with no rights. The farmers were not allowed to grow food, because the British forced them to grow cotton and indigo. The cotton was shipped to Britain and it was used to produce cloths. The Indian people had to pay high taxes for these clothes \cite{friedenFragen2014} and so the Indian people in their own country were restricted and controlled in their way of life. Many laws and restrictions avoided living an equal life and the country and its people were exploited by the British. 

Gandhi had always known how he wanted to see his country and so in 1916 Gandhi made his beliefs and plans public, when the Hindu-University was opened. Gandhi gave a remarkable speech about his idea of India as an autonomic state: In Gandhi's opinion living in Ashrams was the way to achieve his ideas. \cite{hoepken2001gewaltfreiheit} Ashrams were little village in which people live and work but they challenge their life independently as self supporters. He opened his first Ashram a few years ago and had the vision of living there in a kind of "village-democracy". Gandhi at the same time refused the way people lived in India at this time. The urban economic system was not to the way of living he was convicted of. He propagated "Swadeshi" what means "economic contentment (Indians wouldn't need goods from foreign countries) and in this sense also the autarky of India" \cite{hoepken2001gewaltfreiheit}. Therefore the spinning wheel became his sign for India's pursuit of freedom and independence and Gandhi tried to convince the farmers to not make themselves dependent of the English people but to bethink on their traditional farm life instead. For Gandhi this included that farmers should spin their own cotton and make own clothes instead of selling it to the Britain. As a consequence the Indians started an "all-india-spinner's association" and wore their self-made clothes with big honor and pride \cite{friedenFragen2014} followed by a new proud and self-worth within the Indian community.

\paragraph{The Untouchables}
The life of the untouchables was always an issue that kept Gandhi busy all his lifetime. The untouchables were people who were not part of the big four-part caste system (called Varnas) and therefore had fewer rights. The Untouchables were treated inhuman by not talking to them or treating them as dirt. \cite{worldAffairs2014} \todo{check this source. Is this valid?} "They suffered a lot because their jobs were 'not fit' for members of higher castes to perform. They were supposed to be so dirty that it was unholy to even lay a finger on them" \cite{worldAffairs2014}. Gandhi was touched by the destiny of the untouchables and renamed the untouchables as "Harijans", what means "God's Children". His plan was to integrate those people in the community.

\paragraph{Civil Disobedience}
Achieving an independent India in that case meant reacting against the British who had their own goals and principles. Reacting in Gandhi's way meant reacting non-violent like he had fought in South-Africa, too. At the time when Gandhi started realizing his ideas of an independent India, the British government more and more realized that changes had to be made and so they gave some power to the provinces, like education. India was therefore on a good way to become independent. Britain took their time to fulfill the concessions and even planned restrictive laws again, what led to unrest within the Indian population. \cite{agora2014} The British government was more and more afraid of losing control over the Indians and when in 1919 a peaceful meeting of Indians took place, the British \cite{agora2014} carried out a bloodbath where 400 Indians lost their life. Nevertheless, the restrictive laws concerning the Indian population was decided. When in the same year the Indian people made a national strike as a consequence of this bloodbath, the British reacted violently again, but not only the British: also Gandhi's followers used their fists. For Gandhi violence was not acceptable and so this was the finale start of non-cooperation and civil disobedience with Gandhi as it�s leader in India, but he also realized that his nation was not yet ready for his idea of a non-violent movement. He had to go to prison again and saw himself as the guilty party for what had happened and started fasting as sign of penance. \cite{agora2014}

\paragraph{Gandhi's goals in his Movement to Independence}
After Gandhi had left prison in 1924 he was determined to continue more like ever to get back India's independency. He now saw his main task in getting his claims and reforms to the Indian nation. These were in general:
\begin{itemize}
	\item Economic independence (strongly linked with the spinning)
	\item Social unity between Hindus and Muslims and the integration of the Untouchables into society 
\end{itemize}

\paragraph{Salt March}
Apart from getting more rights for the Untouchables, Gandhi's big goal was still to get India independent. Although Gandhi had planned his reforms, which were all attached to this main goal, with accuracy and consideration, his goals and reforms where ignored by the British government. His movement was not going the way he was expecting and wishing it to go. At that point, Gandhi had the idea to point the way by bringing up the salt production. \cite{hoepken2001gewaltfreiheit} The traditional salt production for Indians was paralyzed over the last years because the British had taken over the monopole. Gandhi's plan ended up in the legendary salt marsh, which took place 1930 and lasted 24 days. Thousands of people took part in this marsh or even produced salt themselves. Gandhi forced the British government on his knees and it was from now on allowed to produce Salt for their households, but his big goal was not gained yet.
As a consequence of the salt march, boycotts and the past strikes, India started to become real difficult to govern for Britain. The British were used to treat the Indians as slaves and now the Indians fought back in the name of Satyagrahi, whose number of supporters increased daily. \cite{hungerb�hler1983pioniere} Gandhi's movement achieved worldwide attention and the world was at the same time chocked of how brutal the British reacted against the non-violent movement. 

\paragraph{Britain Invites Gandhi}
The British government seemed themselves in force to react so they invited Gandhi to London in 1931. This round-table-meeting had, apart from smaller concessions, no big results because Britain didn't want to give up their colony that easy and stayed unrelenting. \cite{hungerb�hler1983pioniere} In 1934 Indian's fight for Independence was interrupted again when Gandhi's sign of integrating untouchables into his Ashrams caused big civil unrest. Gandhi still looked out for a way of how his movement could achieve all Indians. He felt certain that a reformation of India could only raise out of the country itself and the people living in it. That was why he traveled around the country to clarify the people about the existing social injustices, but he had little success. Gandhi decided to fight with his own skills and he started fasting to get equal electoral legislation, with success, no one wanted Gandhi to fast till his death. 

\paragraph{The Role of the Indian Parties During the Independence Movement}
Gaining independence for India mainly meant having a political self-government. Years before Gandhi's movement, the Indian National Congress was founded by Hindus and Muslims in 1885. It had, at the beginning, the function to mediate between the Indians and the colonists. \cite{britannica2014} In 1906 the impact of the Hindus in the party increased. As a consequence the Muslims decided to found an own party, called the Muslim League. Later on the National Congress was divided again into a radical and a more moderate part. The moderate part decided to meet every year as an All India Congress Committee. 
In 1916 the National Congress Party and the Muslim League draw a declaration with claims of India becoming independent. It was answered by the British in 1917 by offering more self-government in the future.\todo{(wiki) source?}. 
This early cooperation between the Party and the British colony changed when 1920 Gandhi got the inner spirit leading of the Congress. This was the time when the party changed from a congress to a national movement, who more and more put the British legitimacy in question. \cite{dharampal2010} The National Congress Party from then on combined all aims of the Indian independence movement. \cite{konradin2014}
Consequently, the pressure on the British increased. When in 1937 the Indian National Congress recorded considerable in the provincial elections success, Jawaharlal
Nehru claimed the exclusive representative of the National Congress Party. \cite{dharampal2010} The The British did not offer the Indians the claimed power and even declared Indians entry into the war without their agreement. \cite{dharampal2010} When Gandhi started his quit-india-champagne in 1942, the whole situation started to get out of control. The Indian National Congress offered Britain the choice between leaving the country ultimately or acting in civil disobedience again. The British reacted with merciless military hardness in which about 1000 Indians lost their life. 


\paragraph{India's Independence and Separation into Two States}
When after world war two in 1945 the Labor Party took over the government in Britain, they showed themselves ready to negotiate with the Indians about their independence. \cite{hungerb�hler1983pioniere} India at that time was facing another conflict: this time between Muslims and Hindus. There had never been a major conflict between these two religious group in India so far but when it became clear that the National Congress Party, whose participants were all Hindus at this time, reached more and more independence, this was the time when the voice of the Muslim people in India got louder, demanding an own state: Pakistan. This goal was also spoken out by the participants of the Muslim League, who were afraid of losing their influence. 

In 1946 the past latent existing tension between Muslims and Hindus erupted and lead to violent fights. Gandhi, at that time 77 years old, still traveled through the country to reunite the two groups. He wanted to prevent this separation with all means available, but he could not prevent it in the end. One year later India and Pakistani got their dominion statis and so both were disbanded of British government. This resulted in a mass migration of Muslims moving to Pakistan and Hindus escaping from the new Muslim territory to India. In many villages and cities Hindus and Muslims were fighting against each other.

Gandhi with his remarkable character could not stand the fact that Muslims and Hindu did not patch up. He still had the dream of them living together in harmony. Gandhi's strong will to reunite Hindus and Muslims cost him his life at the end: The father of India, as he was called all over India, was shot by a fanatic Hindu 1948, only half a year later after Indian had become an independent country. 

