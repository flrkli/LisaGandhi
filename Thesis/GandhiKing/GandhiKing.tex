\chapter{Gandhi's Influence on Martin Luther King}
It is not surprising that Gandhi, as the remarkable person he was, had big influences all around the world. One person he influenced was Martin Luther King, like mentioned at the beginning of this seminar paper.

Martin Luther King Jr. was one of the leaders of the Civil Right Movement, who was born into "a system of segregation that separated black people and white people" \cite{manheimer2005martin}. Although slavery was abolished by the 13th amendment after the Civil Right War in 1865 \cite{13amendment}, the white people found new ways to suppress and segregate the black people. \cite{dietrich2008martin} This segregation was enforced by the Jim Crow Laws which were "aimed at denying freedom and equality to the same group that had been enslaved, African Americans." \cite{tischauser2012jim} As a result separated white and black neighborhoods, schools, restaurants and many more developed. This period of segregation, the Jim Crow Era, started after the civil right war in 1881 and lasted until the 1960s \cite{tischauser2012jim}. Only when President Lyndon Johnson signed the Civil Rights Act as a result of the strong Civil Rights Movement in 1964, the Jim Crow Laws became illegal. \cite{tischauser2012jim}

The Civil Rights Movement developed out of the insecurity and suppression of black people at the end of the 19th and the beginning of the 20th century. \cite{dietrich2008martin} Their goal was the emancipation and equality of the blacks. The most important organization to enforce this equality is the National Association for the Advancement of Colored People (NAACP) which was founded in 1909. \cite{dietrich2008martin}

 King experienced the segregation and the enforcement of the Jim Crow Laws from a very early age by growing up in the south of the U.S. and he developed early the idea of an equal society. \cite{manheimer2005martin} Like his grandfather and father, he became a minister \cite{manheimer2005martin} and later he would say that, "the church is my life and I have given my life to the church" \cite{dietrich2008martin}.
 One of his first direct involvements in the Civil Rights Movement was his participation as a spokesman for the Bus Boycott in Montgomery in 1955. This was followed by many campaigns against the segregation of the black people in which he became the voice of the movement. Until his assassination in 1968 he was the leader of the modern American Civil Rights Movement. \cite{aboutKing}

Gandhi's influences on King can be noticed in a direct way by King's various statements that he gave about Gandhi. King also quoted Gandhi in many famous speeches. This direct visible influence will be presented in the first part of the chapter by analyzing different historians and their widespread opinions.
Talking about Gandhi's influences on King also includes an indirect parameter.  There are various similarities that can be drawn between Gandhi's and King's movements and their convictions. These similarities could also be influences of Gandhi on King, but not necessarily. Consequently the second part of the chapter will be more about personal opinions and speculations concerning their similar movement.

\section{Widespread Opinions in Literature}
\paragraph{Montgomery Bus Boycott 1955}
Martin Luther King referred to Gandhi many times in his lifetime. One of these times was during the Montgomery Bus Boycott, which toke place in 1955. The Boycott was about black Americans refusing to use the buses, in which they were separated from the whites.  King took over Gandhi's idea of civil disobedience to show how inhuman and decimating the existing laws were. These laws did not only restrict Blacks behavior in buses but in many other areas of their life, too. Later King said that Gandhi gave him the method, Christ the spirit. \cite{zitelmann2003keiner}

According to Tobias Dietrich \cite{dietrich2008martin}, King had indeed heard of Gandhi before the Bus Boycott but hadn't made any further studies about his life yet. Just when after the Boycott Juliette Hampton compared the Bus Boycott to Gandhi's actions, King picked up this comparison \cite{zitelmann2003keiner} and therefore drew similarities between Gandhi and him by himself, like the author Arnulf Zittelmann asked to bear in mind. 

\paragraph{Kings First Clash with Gandhi's General Idea and How He Found a Point of Orientation in It}
There are various other historians, who describe King as someone who got in contact with Gandhi much earlier during his time at university and not just after Juliette had made the connection between those two movements. \cite{konradinKing2014} These opinions about Gandhi's influence on King will be mentioned in the following.
King visited a course at University in 1948, in which the professor talked about Gandhi. \cite{blakely2001} This was the time when "his warming toward nonviolence began" \cite{blakely2001}.

According to Dietrich, King was even more enthusiastic after he had read some of Gandhi's books. He and lost his initial skepticism about the power of love. Consequently, Kings plan of building a community of various races living in peace was very much inspired by Gandhi's ideas \cite{dietrich2008martin}: "King felt he had found the key by which oppressed people could unlock social protest" \cite{blakely2001} \todo{which source is now the correct one?}. King also emphasized, that he has learned a lot about the social power of love from Gandhi. He was convinced of Gandhi's principle of love as a method of social change and this was one reason for the non-violent Civil Right Movement. \cite{dietrich2008martin}

King himself stated in "Stride Towards Freedom", which was published 1958, about Gandhi's non-violence: "I came to feel this was the only morally and practically sound method open to oppressed people in their struggle for freedom" \cite{sunnemark2003ring}. 
In Zittelmanns opinion, overtaking Gandhi's principle was not enough to make a Gandhi out of King but convinced and confirmed him of acting non-violent \todo{welche quelle? -> ( S. 67 und davor v.a. kein gandhi aus ihm� Keiner dreht mich um)}.
In really acting non-violent, he also stayed behind his Indian idol, according to Dietrich. Gandhi was much more radical and also more integrated in his movement then King ever was \cite{dietrich2008martin}. Critique voices point out that King's references on Gandhi considering their non-violent fight can also be seen as rhetorically clever arguments \cite{dietrich2008martin} to use it in his speeches. 

\paragraph{King Visited India}
King visited India in 1959 and got to know past companions of Gandhi \cite{zitelmann2003keiner}. He also met Nehru, a past fellow companion of Gandhi who was the Prime Minister of India at that time. Nehru who told King to not be naive and uncritical towards Gandhi.  Nehru also told King that follows Gandhi's views in a modified way. \cite{dietrich2008martin} 

Again Critics saw King's naive view of Gandhi as a hint that King just want to see the ideal Gandhi to use his well sounding phrases in his own speeches. \cite{dietrich2008martin} King himself stick to his ideal opinion of Gandhi. Some say, this gave him the benefit of appearing more authentic for his supporters concerning questions of the non-violent movement. \cite{dietrich2008martin} Visiting India also helped him building up strong friendships in India. From then on got primary news about Indians genesis of national emancipation, what possibly helped him with his own fight.
Later he said about this trip that acting non-violent is the strongest weapon of all. \cite{zitelmann2003keiner} Additionally, Martin Luther King tireless underlines the brotherly relationship between Indian people, Africans and Afro-Africans, that all live together in India.  In his speech in the year 1968 he spoke enthusiastically about the "the wonderful spiritual quality of the Indian people" and Gandhi's "beloved community", which could only be achieved by non-violence. \cite{dietrich2008martin}

Led by potential self-doubts King decided to live a simple life after his trip to India, but he gave up this intention by telling that the circumstances in America are different. This retreat of course again reaffirms critiques of how King adopted some of Gandhi's principles just for rhetoric purposes. \cite{dietrich2008martin} Nevertheless, King did not get tired of showing his connection to India, so he once said: "To other countries I may go as a tourist, but to India I come as a pilgrim". \cite{dietrich2008martin}

The author Dietrich sums up King's whole pilgrim to India as not a real one, because he had strongly ideological, political and rhetorical reasons for his trip. \cite{dietrich2008martin} But being in India undoubtedly had one important influence on King's perspective of how social movements should be done: it showed him that it needs time to achieve real changes.

\paragraph{King Grounded Gandhi's Idea in Christianity}
King's program of a non-violent direction continued Gandhi's ideas, but grounded them in a Christian theology to make it more attractive for the western civilization. \cite{dietrich2008martin} For King, Gandhi was an ideal image of Christianity and so he once said: "It is ironic, yet inescapably true that the greatest Christian of the modern world was a man who never embraced Christianity" \cite{blakely2001}. "Gandhism was a way to fight the oppression of black Americans - a method that was consistent with the Christian ethic of love" \cite{blakely2001}. King said that Gandhi probably was the first human being in history, who expanded Jesus ethic of love as something between single individuals on an effective social power on a big scale. \cite{gandhiInfoZentrum2014}

\paragraph{King's Speeches with Quotes of Gandhi's Words}
"As King's career and involvement in a nonviolent struggle went on, his words began to echo Gandhi's own sentiments." \cite{blakely2001} This can be seen as a symbol for King becoming ultimately convinced of Gandhi's principles and he started adopting it on a more personal level then just for tactics. For example, in King's discussion of civil disobedience he once said, "In no sense do I advocate evading or defying the law, as would the rabid segregationist. That would lead to anarchy. One who breaks an unjust law must do so openly, lovingly, and with a willingness to accept the penalty." \cite{blakely2001} Statements like these sound like echoes of what Gandhi had always said. He was also convinced that one should respect the laws, provided it is consistent with "the truth" \cite{blakely2001}. 

To sum up, it seems clear that King was influenced by Gandhi when he regarded non-violence as the strongest weapon of all, emphasized the importance of a beloved community and the power of love. King also learnt from Gandhi how social movements can work out and that he has to be patient to achieve his aims. Gandhi was like an orientation for him and offered him a guideline. 

Critique voices don't deny the influence Gandhi had on King put emphasis that King mainly took over Gandhi's principles to use them for his own benefits like well-sounding speeches.  They therefore assume King to be hypocritical e.g. when he just referred to Gandhi after Juliette did, according to their opinion. Critique voices also add for consideration that visiting India happened for political reasons and did not influence King in the way he always pretended. 

\section{Self-developed Presumptions}
Like mentioned in the chapter before, Gandhi and his beliefs appeared many times during King's political career. Taking over main principles of Gandhi's thoughts showed his excitement and approval towards it. Consequently, whether one sees Kings takeovers of Gandhi's principles as clever steps to rhetorically well speeches or as deeply convinced guideline, there are even more similarities in how both led their movements. These similarities between the two movements will be discussed in the following to present further possible influences.

\paragraph{Unstoppability}
\todo{Unstoppability seems to be no english word!}
Like mentioned before, King's actions were non-violent, too. What is also interesting: both hold on their principles, although that meant for them to go through rough times. This leads to the first similarity: Their unstopppabilty.
For example King was physically attacked for three times, survived three bomb attacks and was in prison for 30 times. \cite{dietrich2008martin} All these experiences did not stop King to go on organizing many sit-ins and boycotts to set an example.  That shows that he was as stubborn and unstoppably as Gandhi was. They were both very convinced and therefore fought extremely passionate and sustainable. King also founded his movement on the basis of love and always against hate. He did not stop to make clear that getting peace has to come from the inside of each human and that was how he moved masses, just like Gandhi did. 
Therefore Gandhi can be seen as a role-model for King and he maybe learnt from him how persistence pays off. Consequently, Gandhi's spirit maybe gave him the enquired endurance during his Civil Right Movement. 

\paragraph{Dreaming}
The most important passive resistance led by King was the march on Washington in 1963, where at the end he hold his famous speech, beginning with "I have a dream�". This famous beginning of the speech led to other connections between both: Both, King and Gandhi, are both linked with the term of dreaming. Like mentioned before, Gandhi's ideals and principles can all be bounded together as him having a huge dream. Also King used the term "dreaming". Maybe again just to sound rhetorically interesting but dreaming was also a part of King's ideology. Their ambitious striving made them to strong and convincing dreamers and only with this character they were able to move all these masses. Gandhi and King as dreaming men can also be seen as a hint for them having a greater goal in mind. Their political actions were both characterized by ideals maybe even utopias. King who got to know Gandhi's dream by studying his books was maybe convicted that a greater goal is needed to succeed. A dream helps the people to understand the direction someone is heading to. 

King's most famous speech also contains many terms that can be linked to Gandhi and Gandhi's dream, too. This helps to draw similarities about how their dreams looked like in detail.
King talked about the black people as "God's people" \cite[line 37]{kingDreamSpeech}, also Gandhi named the Untouchables "Harijans", what means "God's children". This shows how their dreams were linked to their religious beliefs, what will be discussed in a further part of this chapter later on. King wanted to achieve his dream through a strong cohesion by saying: "we will be able to work together, to pray together, to struggle together, to go to jail together, so stand up for freedom together , knowing that we will be free one day" \cite[lines 115-117]{kingDreamSpeech}. King's dream includes a great brotherhood, like Gandhi's did. King learnt during his trip to India that brotherhood and a beloved community is important to achieve changes and that maybe led him to emphasize the social cohesion in his speeches. 
Additionally, Gandhi wanted equality and also King spoke in his famous speech about "that day when all of God's children-black men and white men, Jews and Gentiles, Catholics and Protestants- will be able to join hands" \cite[lines 132-135]. They dreamt their dream of a society without ethical or religious border and distinctions. 
Gandhi's dream was, as mentioned before, directly linked to non-violence as the only real soul force. In King's speech King also spoke about the importance of meeting physical force with soul force \cite[line 54]{kingDreamSpeech}. At that point he stressed out his conviction of non violence as part of his dream. 


To sum up, both men had a deep dream in their mind about how things should look ideally. King maybe took over the symbol of having a dream to reach the people. It has been shown that even the contents of their dreams had several similarities when looking at King's famous speech beginning with "I have a dream".  This can be seen as a further indication for King orientating on Gandhi's beliefs and convictions regarding his general dream. Maybe King maintained parts of what belonged to Gandhi's dream because it had been successful once and King needed some security and self-reassurance in the big fight he was in. Another explanation could be that King was simply convinced of Gandhi's ideas in a very deep way and hoped to help his black nation by follow Gandhi's dream.




