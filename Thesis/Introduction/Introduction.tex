\chapter{Introduction}
"Where there is love there is life" \cite{brainyquoteGandhiLoveLife}. That is only one of Gandhi�s many quotes which are still common used citations and are known worldwide. In a period of war and violence, Gandhi showed the world a new way of fighting, the way of love and non-violence: "Non-violence is the greatest force at the disposal of mankind. It is mightier than the mightiest weapon of destruction devised by the ingenuity of man" \cite{brainyquoteGandhiNonViolance}. Becoming popular as a modest and fragile-looking man, Gandhi went down in history as someone who insisted on his dream with all his power and even life. 

There are not many people that have characterized the 20th century the way Gandhi did. His movement, which in the end made India independent, is known all over the world and even had impacts years after his death. Apart from others, he fascinated Nelson Mandela, Martin Luther King and the Dalai Lama, who according to their own statements overtook many of his principles. In view of the fact that this seminar paper is written for a seminar that deals with the US in the 1960, this work focuses on Martin Luther King as someone whose political work took place in the 60s. 
Therefore this seminar paper deals with the question: How Gandhi�s life, spirit and dream influenced Martin Luther King during the Civil Right Movement in the 1960s?

This question will be answered by first analyzing Gandhi�s life and dream in detail, to understand the way Gandhi lived his dream. Afterwards Gandhi�s obvious influences on King will be depicted by analyzing the view of different authors and King�s direct references about Gandhi. Furthermore in the last chapter self-developed presumptions should explain further possible influences. These possible influences are based on similarities between Gandhi�s and King�s movement. At the end a short Resume will be given to answer the question.