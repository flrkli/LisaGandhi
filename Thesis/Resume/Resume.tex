\chapter{R�sum�}
In this seminar paper the question was asked how Gandhi's life, spirit and dream influenced Martin Luther King. This resume will sum up the past chapters to find a general answer to that question.
Gandhi played the most important role in the Indian independent movement but also helped the Indians in South-Africa to gain more rights. He developed certain principles through his lifetime. Therefore Gandhi's movement was characterized by simplicity and a "moral fasting", but the main principles were non-violence and civil disobedience.  
Discussing the question about how Gandhi influenced King, it turned out that Martin Luther King took over some of Gandhi's main principles and adapted them to the situation in the US at that time. He was convinced of Gandhi's principle of acting non-violent, the importance of the power of love. Therefore King took over methods like boycotts that were similar to Gandhi's civil disobedience. King therefore pursued Gandhi's strategy of demonstrating the Government their own inhuman laws by not taking part in the system any more. King also learnt a lot about how changes can be made and that it needs time to achieve real deep changes. 
The fact that King's speeches sometimes seemed to be copies of Gandhi's can also be explained by their similar made experiences. When King started his non-violent movement he got to know the advantages and the deeper sense of non-violence and these self-made experiences made his speeches sound similar to Gandhi's, who experienced kind of the same.  It all shows that Gandhi had a great influence on King's movement in various ways. What was not answered in this seminar paper was in what extend King was truly convinced in Gandhi's beliefs or if he just used it for his own goals. But his goals were of good nature: he wanted to gain equality for the black people in America.

Also self-developed similarities were presented to make presumptions about how King was maybe influenced furthermore. King's and Gandhi's unstoppability \todo{check word} was addressed and Gandhi as King's potential role-model of never giving up. Additionally, both personalities were described as dreaming persons and King as someone who maybe used the term "dream" to be as successful with his aims as Gandhi was, who was in public always linked to a certain dream. Furthermore, big similarities regarding the content of both dreams were found what can be seen as a further indication for King's orientation on Gandhi or even his great fundamental conviction that Gandhi's way is the only right and true way.  Gandhi possibly encouraged King to fight for what he believed in.
Gandhi's influences on King clarify how big and important Gandhi's movement was, not only for India but also for the rest of the world. Both men, Gandhi and King were desperately fighting for a free world for disadvantaged to create equality and justice around the world. Although both were, apart from similarities and influences, fighting their fights partly different, their movements make both to Pioneers for peace.
