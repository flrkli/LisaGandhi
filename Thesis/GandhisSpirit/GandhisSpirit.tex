\chapter{Gandhi's Spirit and Principles}
Gandhi's movements were both, in India and in South-Africa, influenced by his fundamental principles. These principles bounded together can be seen as his spirit and therefore living these principles can be regarded as Gandhi's dream. 
As a result, Gandhi was not only a man of goals, he was a constantly dreaming person, so all his aims always had a bigger idea in a way of an ideal conception, some may also call it an utopia. Gandhi as a dreaming person should not be understood derogative, because he was not that kind of typical daydreamer. His principles were always carefully thought over not groundless at all. In the following, the grounding of three main principles, which accompanied his movement, will be discussed. 
\begin{itemize}
	\item Fasting (as a way to bear misery and show self-control)
	\item The truth (and how this is linked to the non-violent movement and Civil disobedience)
	\item Simplicity (as Gandhi's way of life)
\end{itemize}

\section{Feasting in Gandhi's Life}
Fasting was always a part of Gandhi's life, as fasting was a very important religious ritual for Gandhi's mother and it gained importance during his non-violent movement. This fact also shows that religion had a big influence on how Gandhi lived. Religion for Gandhi is building the own character and can therefore bring all humans together in peace.\cite{feiler2004} He also saw Hinduism as the most tolerant religion, but was opened for any other religion, too. Gandhi he did not see religion as only rituals and living traditions, he more saw the spiritual value in it and how it can serve community. 
Additionally, Gandhi saw the human being as a "complete unity and integrity of body, mind and soul in the individual human being" \cite{santhanam}. Consequently "he was never tired of saying that the body should be controlled by the mind and the mind by the soul" \todo{source?}. Fasting for Gandhi therefore was also linked with self-control and Gandhi himself emphasized, that one can just heart you if you allow him to. "No one can hurt you without your permission" \todo{check the source. does not really look serious} (http://alles-schallundrauch.blogspot.de/2009/10/gandhis-10-weisheiten-um-die-welt-zu.html). Consequently it is someone's own decision to get hurt or not and how you then deal with injuries and fasting trains this self-control.

As mentioned before, Gandhi also fasted to do penance, apart from other times, when in 1923 his own supporters were fighting in a violent way against the British. Consequently, he saw fasting also as a sign of protest, so it can directly be seen as a part of his non-violent action. Fasting for Gandhi was also self-cleaning after someone failed of convicting the opponent. It therefore should disclose the own deficit concerning the Satyaghara. \cite{gandhiserve}
Gandhi's fasting in 1923 did not have the goal to get attention for the Harijans, how Gandhi called the untouchables. He did not have the aim of utopian agreement between both casts, but he rather wanted the Hindus to understand the hardship which the untouchables have to suffer \cite{gandhiserve} At the fifth day of Gandhi's fasting, when his power started to disappear continuously, representatives of the caste-Hindus and untouchables signed a pact that was a combined electorate procedure but reserved seats in parliament for the untouchables. Finally, the British government also accepted the pact and so Gandhi stopped fasting on his sixth day. This way of fasting is also called the epic fasting, because this emphasize the big extend of fasting till death. \cite{gandhiserve} It therefore improved living conditions of the Harijans in many ways, so Gandhi achieved his goal: While he was fasting, the Hindus were all so concerned about him, that they moved together. All over India temples were opened for Harijans. "The Hindu religion reformed themselves just in these few days, caused be the charisma of Gandhi" \todo{source?}. Gandhi�s fasting therefore changed a religious duty into a more moral duty. \cite{gandhiserve} 
Fasting in Gandhi's way is also connected with doing instead of just preaching, what can be seen as an important characterization of his whole movement. He was not telling the mass what to do, he gave them his life as an example of all his maxims. He also once said: "You must be the change you wish to see in the world" \cite{brainyquoteGandhiChangeWorld}, what again shows that he wanted to be active himself and be a role model for what he stands for. So his opinion was that if you want to change the things around, you have to start by yourself. 

\section{His view of truth, non-violence and Civil Disobedience}
Gandhi himself published a book called "The story of my experiments with the truth". Truth was a significant part of his ideology. It was even in the name of his non-violent organization Satyagraha, like mentioned above. The name of his movement Satyagraha is a mix of two words in Hindi "Truth" on the one hand and "hold on to the truth" on the other hand. \cite{hungerb�hler1983pioniere} Truth for Gandhi can be described as self realization, where path and goal would comply \cite{hoepken2001gewaltfreiheit} like an ideal situation in which things are the way they meant to be. Knowledge of the truth could in his opinion only be received through the service on community. \cite{hoepken2001gewaltfreiheit} 
Holding on the truth was in Gandhi's opinion directly attached to love, because the power of truth was also the power of love. For Gandhi this strength of holding on to the truth was the idea of power. \cite{hungerb�hler1983pioniere} His movement Satyagraha therefore means: power that was born out of truth and carried by the lovers of non-violence. 

At that point it is important to note that Gandhi himself was influenced by many human beings who gave him an orientation of how his movement could look like, too. Especially his non-violence view was influences by past famous people like Thoreau or Socrates. Consequently, it was not only his self-made experiences that led to his principles. Influences that were important for Gandhi's life will not discussed extensively in this work, because this explanations would take us too far away from answering our question. 

Gandhi's meaning of Satyagahra does not mean non-violence as something weak. For him this way of fighting was made out of the human's strength of mind. \cite{hungerb�hler1983pioniere} Acting the non-violent movement he had the goal to show the "rival, that there is a way of non-fighting and it works, too." \cite{santhanam} The goals of Satyagraha was to reach the opponents' conscience by acting non-violent and showing the willingness of even standing physical pain.  This should never happen like an appeal, but rather convince the opponent of a non-violent way and on this way make him a close associative. In contrast, if violence is used this would just cause to more violence and additionally Gandhi was sure that what is gained by violence while always need more violence to exist. This is why all social action should be governed by the same simple set of moral values, of which the main elements are selflessness, non-attachment, nonviolence and active service. \cite{santhanam} Being human and moral was linked together with his view of that every person is the same; there should not be any differences between races or other ethnic groups.  
Gandhi as a well-educated man was of course also influences by other man, like mentioned above Socrates.

Another principle of Gandhi, which was directly linked to the non-violent movement, was the civil disobedience. This principle he took over from Thoreau, which for his part fought against slavery, more precisely against the oppressive laws. The disobedience therefore was addressed to one particular Government and their laws, which was seen as corrupt, inhuman or unfair by the disobedient people.

Gandhi's way of non-violent and non-cooperative resistance was influenced by many western pioneers but it also fit to South-Asian traditions. This is why his movement was that successful and why he reached that many people with his ideas. \cite{dharampal2010}

\section{Gandhi's Simplicity}
Gandhi himself did not have any private property at the end of his life. He refused to have any property during his movement \todo{(ebook anderes als Korntal)} and was only wearing cloths, which for Western people may look more like a scarf. One of these clothes was the Dhoti which was of course made by Indian cotton spinners. This was part of how Gandhi saw life: you don't need much to be happy and to be human \todo{(ebook anderes als Korntal)}. Like mentioned before, Gandhi also refused all kind of materialism, with which the Indian started to live with as a British colony. He wanted them to restart living a simple life like he did. The Ashrams, where Indians should live in a traditional and original way, can be seen as a symbol of Gandhi's simplicity. 


